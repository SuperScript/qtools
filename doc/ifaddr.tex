\documentclass{book}
\usepackage{sstdef}
\title{ifaddr}
\begin{document}
\section{The \cmd{ifaddr} program}

\subsection{Interface}
\begin{code}%
  ifaddr [ \var{field} ... : ] [ \var{addr} ... ]
\end{code}
where \cvar{field} is an 822-format header field and \cvar{addr} is an email
address.  If no \cvar{field} arguments appear, then \cmd{ifaddr} uses
\cmd{To} and \cmd{Cc}.

\cmd{ifaddr} reads a message from standard input, exiting 0 if any \cvar{addr}
appears in a \cvar{field} header, and exiting 100 if there is no match.  Invoked
with no \cvar{addr} arguments, \cmd{ifaddr} looks for the address in the
environment variable \cmd{\$RECIPIENT}.  Address comparisons are case
insensitive.

If it encounters a temporary error while reading input, \cmd{ifaddr} exits 111.

If an address begins with \cmd{@}, then \cmd{ifaddr} compares it to the domain
portion of addresses from each relevant field.


For example, the \cmd{.qmail} line
\begin{code}%
  | condtomaildir \var{dir} ifaddr to cc from : \var{myfriend}
\end{code}
writes correspondence with \cvar{myfriend} to \cvar{dir}.

The invocation
\begin{code}%
  | condtomaildir \var{dir} ifaddr @example.com
\end{code}
writes a message with a \cmd{To} or \cmd{Cc} address in the domain
\cmd{example.com} to \cvar{dir}.
\end{document}
