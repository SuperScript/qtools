%& sstdoc
\banner{qtools}
\begin{document}
\image{../sstlogo.gif}{sstlogo}
\parent{../superscript}{\SST}
\parent{../software}{Software}
\parent{intro}{qtools}
\chapter{The \cmd{ifaddr} program}

\section{Interface}
\begin{code}
  ifaddr [ \arg{field} ... : ] [ \arg{addr} ... ]
\end{code}
where \carg{field} is an 822-format header field and \carg{addr} is an email
address.  If no \carg{field} arguments appear, then \cmd{ifaddr} uses
\bffont{To} and \bffont{Cc}.

\cmd{ifaddr} reads a message from standard input, exiting 0 if any \carg{addr}
appears in a \carg{field} header, and exiting 100 if there is no match.  Invoked
with no \carg{addr} arguments, \cmd{ifaddr} looks for the address in the
environment variable \cmd{\$RECIPIENT}.  Address comparisons are case
insensitive.

If it encounters a temporary error while reading input, \cmd{ifaddr} exits 111.

If an address begins with \cmd{@}, then \cmd{ifaddr} compares it to the domain
portion of addresses from each relevant field.


For example, the \cmd{.qmail} line
\begin{code}
  | condtomaildir \arg{dir} ifaddr to cc from : \arg{myfriend}
\end{code}
writes correspondence with \carg{myfriend} to \carg{dir}.

The invocation
\begin{code}
  | condtomaildir \arg{dir} ifaddr @example.com
\end{code}
writes a message with a \bffont{To} or \bffont{Cc} address in the domain
\cmd{example.com} to \carg{dir}.
\end{document}
