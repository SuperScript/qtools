\documentclass{book}
\usepackage{sstdef}
\title{condtomaildir}
\begin{document}
\section{The \cmd{condtomaildir} program}

\subsection{Interface}
In \cmd{.qmail}:
\begin{code}%
  | condtomaildir \var{dir} \var{prog}
\end{code}
where \cvar{dir} is a Maildir-format directory and \cvar{prog} is one
or more arguments specifying a program to run for each message.

For each email message it processes, \cmd{condtomaildir} runs
\cvar{prog} with the message on standard input.  If \cvar{prog} exits
0, \cmd{condtomaildir} writes the email message to \cvar{dir}, and
then exits 99, so that further commands in \cmd{.qmail} are ignored.
If \cvar{prog} exits 111, \cmd{condtomaildir} exits 111, so delivery
is retried later. If \cvar{prog} exits anything else, or does not
exist, \cmd{condtomaildir} exits 0, so the rest of \cmd{.qmail} is
processed.

If it encounters a permanent error in handling a message,
\cmd{condtomaildir} exits 100.  If it encounters a temporary
error in handling a message, \cmd{condtomaildir} exits 111.


If \cmd{condtomaildir} encounters a permanent error attempting to run
\cvar{prog}, it exits 100.  If \cmd{condtomaildir} encounters a
temporary error in attempting to run \cvar{prog}, it exits 111.
\end{document}
