%& sstdoc
\banner{qtools}
\begin{document}
\image{../sstlogo.gif}{sstlogo}
\parent{../superscript}{\SST}
\parent{../software}{Software}
\parent{intro}{qtools}
\chapter{The \cmd{condtomaildir} program}

\section{Interface}
In \cmd{.qmail}:
\begin{code}
  | condtomaildir \arg{dir} \arg{prog}
\end{code}
where \carg{dir} is a Maildir-format directory and \carg{prog} is one
or more arguments specifying a program to run for each message.

For each email message it processes, \cmd{condtomaildir} runs
\carg{prog} with the message on standard input.  If \carg{prog} exits
0, \cmd{condtomaildir} writes the email message to \carg{dir}, and
then exits 99, so that further commands in \cmd{.qmail} are ignored.
If \carg{prog} exits 111, \cmd{condtomaildir} exits 111, so delivery
is retried later. If \carg{prog} exits anything else, or does not
exist, \cmd{condtomaildir} exits 0, so the rest of \cmd{.qmail} is
processed.

If it encounters a permanent error in handling a message,
\cmd{condtomaildir} exits 100.  If it encounters a temporary
error in handling a message, \cmd{condtomaildir} exits 111.


If \cmd{condtomaildir} encounters a permanent error attempting to run
\carg{prog}, it exits 100.  If \cmd{condtomaildir} encounters a
temporary error in attempting to run \carg{prog}, it exits 111.
\end{document}
