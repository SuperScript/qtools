%& sstdoc
\banner{qtools}
\begin{document}
\image{../sstlogo.gif}{sstlogo}
\parent{../superscript}{\SST}
\parent{../software}{Software}
\parent{intro}{qtools}
\chapter{The \cmd{filterto} program}

\section{Interface}
In \cmd{.qmail}:
\begin{code}
  | filterto \arg{address} \arg{prog}
\end{code}
where \carg{address} is an email address, and \carg{prog} is one or
more arguments specifying a program to run for each message.

For each email message it processes, \cmd{filterto} runs \carg{prog}
with the message on standard input.  If \carg{prog} exits 0,
\cmd{filterto} forwards the standard output from \carg{prog} to
\carg{address}, and then exits 99, so that further commands in
\cmd{.qmail} are ignored.  If \carg{prog} exits 111, \cmd{filterto}
exits 111, so delivery will be retried later.  If \carg{prog} exits
with any other exit code, or does not exist, \cmd{filterto} exits 0,
so the rest of \cmd{.qmail} is processed.

Note that it is not safe for \carg{prog} to fork a child that reads
the message in the background.

\bffont{NB}: If you use \cmd{filterto} in a \cmd{.qmail} file, make
sure to add a line specifying delivery to your normal mailbox.
\end{document}
