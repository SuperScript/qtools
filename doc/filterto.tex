\documentclass{book}
\usepackage{sstdef}
\title{filterto}
\begin{document}
\section{The \cmd{filterto} program}

\subsection{Interface}
In \cmd{.qmail}:
\begin{code}%
  | filterto \var{address} \var{prog}
\end{code}
where \cvar{address} is an email address, and \cvar{prog} is one or
more arguments specifying a program to run for each message.

For each email message it processes, \cmd{filterto} runs \cvar{prog}
with the message on standard input.  If \cvar{prog} exits 0,
\cmd{filterto} forwards the standard output from \cvar{prog} to
\cvar{address}, and then exits 99, so that further commands in
\cmd{.qmail} are ignored.  If \cvar{prog} exits 111, \cmd{filterto}
exits 111, so delivery will be retried later.  If \cvar{prog} exits
with any other exit code, or does not exist, \cmd{filterto} exits 0,
so the rest of \cmd{.qmail} is processed.

Note that it is not safe for \cvar{prog} to fork a child that reads
the message in the background.

\textbf{NB}: If you use \cmd{filterto} in a \cmd{.qmail} file, make
sure to add a line specifying delivery to your normal mailbox.
\end{document}
