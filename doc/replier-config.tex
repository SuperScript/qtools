%& sstdoc
\banner{qtools}
\begin{document}
\image{../sstlogo.gif}{sstlogo}
\parent{../superscript}{\SST}
\parent{../software}{Software}
\parent{intro}{qtools}
\chapter{The \cmd{replier-config} Program}

\section{Interface}
\begin{code}
  replier-config \arg{dir} \arg{dot} \arg{local} \arg{host} [ \arg{outlocal} [ \arg{outhost} ]]
\end{code}
where \carg{dir} is a directory, \carg{dot} is a \cmd{.qmail} file,
\carg{local} is the local portion of an email address, and \carg{host}
is a host name, \carg{outlocal} is either the local portion of an
email address or the empty string, and \carg{outhost} is a host name.
The directory \carg{dir} must begin with a slash.
If \carg{outlocal} is not present, \cmd{replier-config} sets it to
\carg{local}.  If \carg{outhost} is not present, \cmd{replier-config}
sets it to \carg{host}.

If \carg{outlocal} is not empty, \cmd{replier-config} creates a new
email replier at the address
\begin{code}
  \arg{outlocal}-help@\arg{outhost}
\end{code}

If \carg{outlocal} is empty, \cmd{replier-config} creates a new email
replier at the address
\begin{code}
  help@\arg{outhost}
\end{code}

In either case, \cmd{replier-config} sets up two \cmd{.qmail} files to
control the help-address replier: \cmd{\arg{dot}-help} and
\cmd{\arg{dot}-return-default}.  These files should control messages
to the addresses \cmd{\arg{outlocal}-help@\arg{outhost}} and
\cmd{\arg{outlocal}-return-@\arg{outhost}}, respectively (or
\cmd{help@\arg{outhost}} and \cmd{return-@\arg{outhost}} if
\carg{outlocal} is empty).

Within \carg{dir}, \cmd{replier-config} creates several configuration
files for \cmd{replier}.

The first line of \cmd{\arg{dir}/inlocal} consists of \carg{local}.

The first line of \cmd{\arg{dir}/inhost} consists of \carg{host}.

The first line of \cmd{\arg{dir}/outlocal} consists of \carg{outlocal}.

The first line of \cmd{\arg{dir}/outhost} consists of \carg{outhost}.

The first line of \cmd{\arg{dir}/mailinglist} consists of 
\cmd{contact \arg{helpaddress}; run by replier}, where
\carg{helpaddress} is the help address created by
\cmd{replier-config} as described above.

The \cmd{\arg{dir}/headerremove} file contains
\begin{code}
  return-path
  return-receipt-to
  content-length
\end{code}

The \cmd{\arg{dir}/headeradd} file contains
\begin{code}
  Precedence: bulk
\end{code}

The \cmd{\arg{dir}/text/help} file contains text sent in response to
messages received at the help address.

\section{Typical usage}
To configure a replier at the address \cmd{joe-replier-help@example.com}, the
user \cmd{joe} issues the command
\begin{code}
  replier-config \swungdash /replier \swungdash /.qmail-replier joe-replier example.com
\end{code}

If email to \cmd{joe@example.com} is delivered to
\cmd{joe@joehost.example.com}, the following command may be
appropriate to configure a replier at
\cmd{joe-replier-help@example.com}:
\begin{code}
   replier-config \swungdash /replier \swungdash /.qmail-replier joe-replier joehost.example.com joe-replier example.com
\end{code}

If the user \cmd{replier} handles mail for the virtual domain
\cmd{replier.example.com}, then the following command creates a
replier at \cmd{help@replier.example.com}:
\begin{code}
   replier-config \swungdash /replier \swungdash /.qmail replier example.com ""
\end{code}

\section{Adding replier addresses}
To add a new replier address that filters only the body of a message,
edit the \cmd{\arg{dir}/bodyfilter} shell script and add a new case
corresponding to the new address.  Then create a symbolic link from
\cmd{\arg{dir}/qmail-bodyfilter} to the \cmd{.qmail} file that
controls the new address.  That's it!

Adding a command to filter the message header is entirely analagous,
but use \cmd{\arg{dir}/headerfilter} and
\cmd{\arg{dir}/qmail-headerfilter}, while entire-message filters use
\cmd{\arg{dir}/msgfilter} and \cmd{\arg{dir}/qmail-msgfilter}.


\section{Bounce handling}
By default, a replier created with \cmd{replier-config} silently
throws away bounce messages.  The \cmd{.qmail} file controlling
bounces is a symbolic link pointing to \cmd{\arg{dir}/bouncer}.  Edit
\cmd{\arg{dir}/bouncer} to change the treatment of bounce messages.

\end{document}
