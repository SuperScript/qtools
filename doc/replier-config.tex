\documentclass{book}
\usepackage{sstdef}
\title{replier-config}
\begin{document}
\section{The \cmd{replier-config} Program}

\subsection{Interface}
\begin{code}%
  replier-config \var{dir} \var{dot} \var{local} \var{host} [ \var{outlocal} [ \var{outhost} ]]
\end{code}
where \cvar{dir} is a directory, \cvar{dot} is a \cmd{.qmail} file,
\cvar{local} is the local portion of an email address, and \cvar{host}
is a host name, \cvar{outlocal} is either the local portion of an
email address or the empty string, and \cvar{outhost} is a host name.
The directory \cvar{dir} must begin with a slash.
If \cvar{outlocal} is not present, \cmd{replier-config} sets it to
\cvar{local}.  If \cvar{outhost} is not present, \cmd{replier-config}
sets it to \cvar{host}.

If \cvar{outlocal} is not empty, \cmd{replier-config} creates a new
email replier at the address
\begin{code}%
  \var{outlocal}-help@\var{outhost}
\end{code}

If \cvar{outlocal} is empty, \cmd{replier-config} creates a new email
replier at the address
\begin{code}%
  help@\var{outhost}
\end{code}

In either case, \cmd{replier-config} sets up two \cmd{.qmail} files to
control the help-address replier: \cmd{\var{dot}-help} and
\cmd{\var{dot}-return-default}.  These files should control messages
to the addresses \cmd{\var{outlocal}-help@\var{outhost}} and
\cmd{\var{outlocal}-return-@\var{outhost}}, respectively (or
\cmd{help@\var{outhost}} and \cmd{return-@\var{outhost}} if
\cvar{outlocal} is empty).

Within \cvar{dir}, \cmd{replier-config} creates several configuration
files for \cmd{replier}.

The first line of \cmd{\var{dir}/inlocal} consists of \cvar{local}.

The first line of \cmd{\var{dir}/inhost} consists of \cvar{host}.

The first line of \cmd{\var{dir}/outlocal} consists of \cvar{outlocal}.

The first line of \cmd{\var{dir}/outhost} consists of \cvar{outhost}.

The first line of \cmd{\var{dir}/mailinglist} consists of 
\cmd{contact \var{helpaddress}; run by replier}, where
\cvar{helpaddress} is the help address created by
\cmd{replier-config} as described above.

The \cmd{\var{dir}/headerremove} file contains
\begin{code}%
  return-path
  return-receipt-to
  content-length
\end{code}

The \cmd{\var{dir}/headeradd} file contains
\begin{code}%
  Precedence: bulk
\end{code}

The \cmd{\var{dir}/text/help} file contains text sent in response to
messages received at the help address.

\subsection{Typical usage}
To configure a replier at the address \cmd{joe-replier-help@example.com}, the
user \cmd{joe} issues the command
\begin{code}%
  replier-config \swungdash /replier \swungdash /.qmail-replier joe-replier example.com
\end{code}

If email to \cmd{joe@example.com} is delivered to
\cmd{joe@joehost.example.com}, the following command may be
appropriate to configure a replier at
\cmd{joe-replier-help@example.com}:
\begin{code}%
   replier-config \swungdash /replier \swungdash /.qmail-replier joe-replier joehost.example.com joe-replier example.com
\end{code}

If the user \cmd{replier} handles mail for the virtual domain
\cmd{replier.example.com}, then the following command creates a
replier at \cmd{help@replier.example.com}:
\begin{code}%
   replier-config \swungdash /replier \swungdash /.qmail replier example.com ""
\end{code}

\subsection{Adding replier addresses}
To add a new replier address that filters only the body of a message,
edit the \cmd{\var{dir}/bodyfilter} shell script and add a new case
corresponding to the new address.  Then create a symbolic link from
\cmd{\var{dir}/qmail-bodyfilter} to the \cmd{.qmail} file that
controls the new address.  That's it!

Adding a command to filter the message header is entirely analagous,
but use \cmd{\var{dir}/headerfilter} and
\cmd{\var{dir}/qmail-headerfilter}, while entire-message filters use
\cmd{\var{dir}/msgfilter} and \cmd{\var{dir}/qmail-msgfilter}.


\subsection{Bounce handling}
By default, a replier created with \cmd{replier-config} silently
throws away bounce messages.  The \cmd{.qmail} file controlling
bounces is a symbolic link pointing to \cmd{\var{dir}/bouncer}.  Edit
\cmd{\var{dir}/bouncer} to change the treatment of bounce messages.

\end{document}
