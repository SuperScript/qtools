%& sstdoc
\banner{qtools}
\begin{document}
\image{../sstlogo.gif}{sstlogo}
\parent{../superscript}{\SST}
\parent{../software}{Software}
\parent{intro}{qtools}
\chapter{The \cmd{checkdomain} program}

\section{Interface}
\begin{code}
  checkdomain \arg{dom} ...
\end{code}
where \carg{addr} is an email header name.

\cmd{checkdomain} reads email addresses from standard input,
and compares the domain portion with each \carg{dom}
without regard to case.
If no \carg{addr} appears, \cmd{checkdomain} uses the domain
part of the environment variable \cmd{\$RECIPIENT}
as \carg{dom}.  If it finds a match, \cmd{checkdomain} exits 0.
Otherwise, \cmd{checkdomain} exits 100.

\cmd{checkdomain} reads addresses in the form output by 
\href{\cmd{822addr}}{822addr.html}.  Each address appears
as a null-terminated line with an introductory character.
A \cmd{+} character introduces a
delivery address, and a \cmd{(} character introduces a comment.
The introductory characters are not part of the address.
\cmd{checkdomain} compares its arguments only with delivery
addresses, and ignores comments.

If \cmd{checkdomain} finds an address matching \carg{dom}, it exits 0.
Otherwise, \cmd{checkdomain} exits 100.
\end{document}
