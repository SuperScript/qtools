\documentclass{book}
\usepackage{sstdef}
\title{checkaddr}
\begin{document}
\section{The \cmd{checkaddr} program}

\subsection{Interface}
\begin{code}%
  checkaddr \var{addr} ...
\end{code}
where \cvar{addr} is an email header name.

\cmd{checkaddr} reads email addresses from standard input,
and compares them with each \cvar{addr} without regard to case.
If no \cvar{addr} appears, \cmd{822addr} uses
\cmd{\$RECIPIENT} as \cvar{addr}.
If it finds a match, \cmd{checkaddr} exits 0.  Otherwise,
\cmd{checkaddr} exits 100.

\cmd{checkaddr} reads addresses in the form output by 
\href{\cmd{822addr}}{822addr.html}.  Each address appears
as a null-terminated line with an introductory character.
A \cmd{+} character introduces a
delivery address, and a \cmd{(} character introduces a comment.
The introductory characters are not part of the address.
\cmd{checkaddr} compares its arguments only with delivery
addresses, and ignores comments.

If \cmd{checkaddr} finds an address matching \cvar{addr}, it exits 0.
Otherwise, \cmd{checkaddr} exits 100.

\end{document}
